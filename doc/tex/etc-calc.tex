\documentclass[11pt]{article}

\usepackage[margin=1.3in]{geometry}
\usepackage{amsmath}
\usepackage{graphicx}

\title{DESI Online ETC Forecasts\\
{\em\Large Version 0.2}}
\author{David Kirkby and Anze Slosar}

\providecommand{\eqn}[1]{eqn.~(\ref{eqn:#1})}
\providecommand{\tab}[1]{Table~\ref{tab:#1}}
\providecommand{\fig}[1]{Figure~\ref{fig:#1}}

%\providecommand{\vecsymbol}[1]{\ensuremath{\boldsymbol{#1}}}
%\providecommand{\Dv}{\vecsymbol{D}}

\begin{document}
\maketitle

\section{Introduction}

We wish to estimate the signal-to-noise ratio $\nu = S/N$ of a spectrum by adding a sequence of independent estimates of the spectrum's accumulated signal $S_i$ and background $B_i$ (mostly sky) during each iteration of the telescope guiding loop,
\begin{equation}
S = \sum_i S_i \quad, \quad B = \sum_i B_i \quad , \quad N^2 = S + B \; ,
\end{equation}
where $i$ indexes the guider exposures taken at $\simeq 1$~Hz during a $\simeq 20$~minute spectroscopic integration. $S$ and $B$ represent some integrated signal and background metrics in units of electrons, appropriately defined for some nominal spectral target. The resulting variance of the estimated signal-to-noise ratio is
\begin{equation}
\frac{\sigma_\nu^2}{\nu^2} = \frac{r^2}{4} \left(\frac{\sigma_B}{B}\right)^2 + \frac{(1+r)^2}{4} \left(\frac{\sigma_S}{S}\right)^2
\quad , \quad r \equiv \frac{B}{S+B}
\; ,
\end{equation}
with
\begin{equation}
\sigma_{S}^2 = \sum_i \sigma_{S_i}^2 \quad , \quad \sigma_{B}^2 = \sum_i \sigma_{B_i}^2 \; .
\end{equation}
If conditions are not changing during a spectroscopic exposure, then $\sigma_{\nu}/\nu$ scales with $n^{-1/2}$, where $n$ is the number of elapsed guide exposures, and this will still be a good approximation when conditions are changing gradually. Therefore, in the background-dominated limit (for the spectroscopic target), $r \rightarrow 1$ and
\begin{equation}
\frac{\sigma_\nu^2}{\nu^2} \simeq \frac{1}{n}\left[
\frac{1}{4} \left(\frac{\sigma_B}{B}\right)^2 + \left(\frac{\sigma_S}{S}\right)^2\right] \; .
\label{eqn:snr-bglim}
\end{equation}

Assume that the relevant atmospheric and sky properties are uniform across the focal plane, then the time-dependence of $S_i$ can then be modeled as
\begin{equation}
S_i = \alpha \epsilon_i A(w_i)
\end{equation}
where $\alpha$ is some time-independent constant, $\epsilon_i$ tracks the changing atmospheric throughput, and $A(w)$ gives the fraction of light entering a fiber given a PSF FWHM size $w$. The corresponding variance is then
\begin{equation}
\frac{\sigma_{S_i}^2}{S_i^2} =
\frac{\sigma_{\epsilon_i}^2}{\overline{\epsilon}_i^2} +
\rho_i\frac{\sigma_{\epsilon_i}}{\overline{\epsilon}_i}\frac{\sigma_{w_i}}{\overline{w}_i} \left( \overline{w}_i\frac{A'}{A}\right) +
\frac{\sigma_{w_i}^2}{\overline{w}_i^2} \left( \overline{w}_i\frac{A'}{A}\right)^2
\label{eqn:sig-error}
\end{equation}
where $\overline{\epsilon}_i$ and $\overline{w}_i$ are the true values for exposure $i$, $A$ and $A'$ are evaluated at $\overline{w}_i$, and $\rho_i$ is the correlation coefficient for the estimated $\epsilon_i$ and $w_i$ (which is generally large and positive).

Each guider exposure $i$ provides an estimated total detected flux $f_i$ (in electrons) and PSF FWHM size $w_i$ (in arcseconds) for $n_\ast \sim 10$ stars.  For simplicity, assume that all guide stars are identical with true $\epsilon = 1$ flux $f$ (in electrons) and take $\epsilon_i = f_i/f$. The guider also provides an estimate $b_i$ of the spatially flat (sky + dark current) background level in electrons per unit guide sensor area per guide exposure, which we take to be uniform across the guide sensors with true value $\overline{b}_i$ so that
\begin{equation}
B_i = \beta (b_i - d(T_i)) + D
\end{equation}
for some time-independent constant $\beta$, assumed dark current model $d(T)$ evaluated at the measured temperature $T_i$ and constant dark current $D$ in electrons per unit spectrograph sensor area per guide exposure. The corresponding variance is
\begin{equation}
\sigma_{B_i}^2 = \beta\left((\overline{b_i}  + \sigma_{\text{ro}}^2)A^{-1} + d'(\overline{T}_i)^2 \sigma_T^2\right) + D
\end{equation}
where $A$ is the total signal-free guide sensor area used to estimate $b_i$, $\sigma_{\text{ro}}^2$ is the readout noise variance, $\overline{T}_i$ is the true temperature and $\sigma_T^2$ is the temperature measurement variance.

\section{Background Estimates}

The e2v ccd230-42 sensor that has been selected for the GFA has a dark current in electrons per pixel per guide exposure of
\begin{equation}
d(T) = 19335 T^3 \exp(-6400/T)\,t_{\text{exp}} \; ,
\end{equation}
with sensor temperature $T$ measured in Kelvin, corresponding to $d(T) \simeq 80$ elec/pix/exp and $d'(T)\sigma_T \simeq 1.4$ elec/pix/exp at 20$^\circ$C with $\sigma_T = 0.1^\circ$C. We assume that $D \ll B_i$, since the spectrograph sensor is cooled, so that the ratio $\sigma_{B_i}/B_i$ is independent of $\beta$ to a good approximation.

\tab{bgpar} summarizes the nominal parameter values we assume for our estimates and \fig{bgpar} shows the corresponding dependence of $\sigma_{B_i}/B_i$ on the area $A$, guide sensor temperature $\overline{T}$ and temperature measurement error $\sigma_T$. Recall from \eqn{snr-bglim} that $(\sigma_{B_i}/B_i)/2$ sets the $\sigma_{\nu}/\nu$ floor for background-dominated spectral targets. The area required to saturate the guide sensor background measurement depends on how accurately the sensor temperature is measured: with $\sigma_T = 0.1^\circ$C, $\sim 1000$ pixels are sufficient but with $\sigma_T = 0.01^\circ$C, an order of magnitude more pixels are required. However, the overall error and temperature sensitivity are greatly reduced with the smaller $\sigma_T$ so it may not be necessary to saturate the background measurement area if $\sigma_T \simeq 0.01^\circ$C can be achieved.

\begin{table}[htb]
\begin{center}
\begin{tabular}{lcr}
Parameter & Symbol & Value \\
\hline
GFA Temperature & $\overline{T}$ & 20$^\circ$C \\
GFA exposure time & $t_{\text{exp}}$ & 0.5 sec \\
GFA dark current & $\overline{d}$ & 80 elec/pix/exp \\
GFA readout noise & $\sigma_{\text{ro}}$ & 20 elec/exp \\
Nominal sky brightness & $\overline{b} - \overline{d}$ & 10.9 elec/pix/exp \\
\hline
\end{tabular}
\end{center}
\caption{Nominal parameters assumed for the background calculations in this section.}
\label{tab:bgpar}
\end{table}

\begin{figure}[htb]
\begin{center}
\includegraphics[width=5in]{bg}
\caption{Predicted fractional error $\sigma_{B_i}/B_i$ for estimating the incremental spectral background during a single guider exposure as a function of the signal-free area $A$ used to estimate the guider background. For comparison, each of the 10 GFA sensors has an active area of 2 Mpix. Solid (dashed) curves show predictions for guide sensor operating temperatures of 10, 20, 30$^\circ$C with measurement errors of $\pm0.1^\circ$C ($\pm 0.01^\circ$C).}
\label{fig:bgpar}
\end{center}
\end{figure}

\section{Gaussian PSF Approximation}

Assume a Gaussian PSF with standard deviation $\sigma$ then $w = 2\sqrt{2\log 2}\,\sigma$. In the limit where $\sigma \gg$ (guide sensor pixel size) and observations are background dominated, $b \sigma^2 \gg f$, we can calculate the expected errors analytically, finding a covariance matrix for measuring $(\epsilon_i,w_i)$ of
\begin{equation}
C_{\ast} = \nu_{\ast,i}^{-2}\begin{pmatrix}
\overline{\epsilon}_i^2 & \overline{\epsilon}_i \overline{w}_i/2\\
\overline{\epsilon}_i \overline{w}_i/2 & \overline{w}_i^2/2
\end{pmatrix}\quad, \quad
\nu_{\ast,i}^2 = \frac{\log 2}{\pi}\, \frac{\overline{\epsilon}_i^2 f^2}{b \overline{w}_i^2}\, n_{\ast}
\end{equation}
where $\nu_{\ast,i}$ is the total SNR per guide exposure for $n_{\ast}$ stars, and we find $\rho_i = 1/\sqrt{2}$. The corresponding RMS centroid measurement errors are uncorrelated with the flux and size estimates
\begin{equation}
\sigma_{xy} = \frac{\overline{w}_i\,\nu_{\ast,i}^{-1}}{2\sqrt{2\log 2}} \; .
\end{equation}
Note that the appropriate $\nu_\ast$ to use here is marginalized over the unknown size of the PSF, leading to smaller values than are obtained with the usual ansatz of an optimal matched filter. The fiber acceptance fraction for a Gaussian PSF is
\begin{equation}
A(w) = 1 - 2^{-D^2/(w^2+w_j^2)}
\end{equation}
where $D = 1.46$ arcseconds is the fiber diameter and we assume that targets are centered on their fiber, on average, with a Gaussian jitter of FWHM $w_j \simeq 0.235$ arcseconds (0.100 arcseconds RMS).

\begin{figure}[htb]
\begin{center}
\includegraphics[width=5in]{sig}
\caption{Predicted fractional error $\sigma_{S_i}/S_i$ for estimating the incremental spectral signal during a single guider exposure as a function of the total guide star SNR per guide exposure assuming a PSF FWHM of 1.2 arcseconds. Curves show PSF FWHM values of 1.6,1.2,0.8 arcseconds and are calculated assuming a 0.1 arcsecond RMS jitter for the telescope point and no average centroid offset. Dashed curves show the improvement due to a 25\% reduction in the sky level relative to the nominal level.}
\label{fig:sigpar}
\end{center}
\end{figure}

\section{Realistic PSF Corrections}

The assumptions behind the Gaussian PSF approximation are not particularly good: the PSF is not Gaussian, the PSF is not well sampled by the guide sensor pixels, and guide star images are not background dominated. In this section, we estimate the corrections due a more realistic PSF model. For our PSF model, we convolve the following three components:
\begin{itemize}
\item an Airy function for a 3.8m diameter mirror with 25\% of its area obscured by the secondary mirror,
\item a Kolmogorov function with FWHM varying from 0.8--1.6 arcseconds, and
\item a Gaussian function with RMS 0.1 arcseconds to model pointing jitter.
\end{itemize}
We include realistic pixel sampling using the GalSim program to render the PSF into a 20 by 20 pixel postage stamp with 15 micron pixels ($\simeq 0.22$ arcsec) and a random sub-pixel offset. Finally, we calculate the covariance matrix numerically as the inverse Fisher matrix, including the signal contribution to the per-pixel variance. The results are summarized in \fig{acceptance} and \fig{covariance}. Our conclusions are that:
\begin{itemize}
\item The fiber acceptance fraction $A(w)$ is reduced by 10--15\% compared with the Gaussian approximation but the change in the relevant combination $w A'(w)/A(w)$ in \eqn{sig-error} is negligible, except for the smallest FHWM values.
\item A offset of up to 200 mas between the PSF centroid and the fiber center has a negligible impact.
\item The expected errors for measuring the PSF FWHM are increased by 10-20\% but corrections to the centroid measurement error and correlation coefficient are less than 5\%.
\end{itemize}

\begin{figure}[htb]
\begin{center}
\includegraphics[width=5.5in]{acceptance}
\caption{Fiber acceptance function $A(w)$ (left) and the combination $w A'(w)/A(w)$ (right) appearing in \eqn{sig-error}. Predictions are for the realistic PSF model and pixel sampling described in the text. Solid curves show the prediction for a PSF that is offset from the fiber center by 0, 100, or 200 mas.}
\label{fig:acceptance}
\end{center}
\end{figure}

\begin{figure}[h]
\begin{center}
\includegraphics[width=4.5in]{covariance}
\caption{Ratios of quantities calculated for the realistic and Gaussian PSF models.  Curves show the Fisher-matrix predictions for the FWHM error $\sigma_w$, the $\epsilon$-$w$ correlation coefficient $\rho$, and the centroid error $\sigma_{xy}$ as a function of the atmospheric PSF FWHM. Predictions for both models scale with the (fully marginalized) SNR $\nu_\ast$ and the Gaussian prediction is indepdent of FWHM.}
\label{fig:covariance}
\end{center}
\end{figure}

\section{Summary}

To summarize we present a few benchmark scenarios to illustrate the range of expected performance:
\begin{enumerate}
\item Our baseline model assumes the following conditions that are constant during the spectroscopic integration: $\nu_{\ast} = 100$, $T = 20^\circ$C, $w = 1.2$ arcsec.
\item We linearly increase the sky level by 20\% over 20 minutes while holding all other parameters fixed at their baseline values. This variation increases $B_i$ and (mildly, via $\nu_{\ast}$) $\sigma_{S_i}$.
\item We linearly increase the guide sensor temperature from 20 to $30^\circ$C over 20 minutes while holding all other parameters fixed at their baseline values. This variation increases $\sigma_{B_i}$ and (mildly, via $\nu_{\ast}$) $\sigma_{S_i}$.
\item We linearly increase the Gaussian PSF FWHM from 1.4 to 1.6 arcseconds over 20 minutes while holding all other parameters fixed at their baseline values. This variation decreases $S_i$ and increases (mildly, via $\nu_{\ast}$) $\sigma_{S_i}/S_i$.
\end{enumerate}
All of our scenarios assume $\sigma_T = 0.1^\circ$C, $A = 10^4$ pixels, in addition to the parameter values in \tab{bgpar}. The time required to reach the target $\nu = 10$ varies from 15.3--17.7 minutes between these scenarios, with completion windows of 3.3--4.3 seconds.

\begin{figure}[htb]
\begin{center}
\includegraphics[width=5.5in]{project}
\caption{Projected evolution of $B_i$ (blue), $4 S_i$ (red) and $\nu$ (green) during a 20-minute spectroscopic integration for the four scenarios described in the text. Note that $B_i$ and $4 S_i$ use the right-hand scale with a suppressed zero.  The factor of 4 scaling applied to $S_i$ was chosen arbitrarily to improve the clarity of the plots. All integrations are tracked to a target SNR of 10. Ranges show $\pm 1\sigma$, including for $\nu$ where the spread is too small to see.}
\label{fig:project}
\end{center}
\end{figure}

Some open issues for possible future investigation:
\begin{itemize}
\item The PSF is wavelength dependent with FWHM varying by about 20\% across the spectrograph, so differences between the guide star SED, convolved with the GFA filter, and the target SED introduce small chromatic effects. This is probably a negligible effect at the level of accuracy required for the online ETC.
\item Targets are generally not centered on their fibers except at the nominal integration midpoint, even with perfect guiding. These effects are calculable but should be smaller than the pointing jitter, so are probably negligible for the online ETC.
\end{itemize}

\end{document}
